\section*{Theoretical examination \modulecode}
The general shape of a theoretical exam for \texttt{DEV 3} is made up of a series of highly structured open questions.

\paragraph{Question 1: } \ \\

\textbf{Concrete example of question:} \textit{}

\ \\ 

\textbf{Concrete example of answer:} \textit{}

\ \\

\textbf{Points:} \textit{25\%.}

\ \\ 

\textbf{Grading:} \textit{}

\ \\ 

\textbf{Associated learning objective:} \texttt{ABS}

\ \\ 

\paragraph{Question 2: } \ \\

\textbf{Concrete example of question:} \textit{}

\ \\

\textbf{Concrete example of answer:} \textit{}

\ \\

\textbf{Points:} \textit{25\%.}

\ \\

\textbf{Grading:} \textit{}

\ \\

\textbf{Associated learning objective:} \texttt{ENC}

\ \\

\paragraph{Question 3: } \ \\

\textbf{Concrete example of question:} \textit{}

\ \\

\textbf{Concrete example of answer:} \textit{}

\ \\

\textbf{Points:} \textit{25\%.}

\ \\

\textbf{Grading:} \textit{}

\ \\

\textbf{Associated learning objective:} \texttt{TYP}

\ \\

\paragraph{Question 4: } \ \\

\textbf{Concrete example of question:} \textit{}

\ \\

\textbf{Concrete example of answer:} \textit{}

\ \\

\textbf{Points:} \textit{25\%.}

\ \\

\textbf{Grading:} \textit{}

\ \\

\textbf{Associated learning objective:} \texttt{BHF}

\ \\

