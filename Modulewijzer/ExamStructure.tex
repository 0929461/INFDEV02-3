\section*{Structure of exam \modulecode}
The general shape of a theoretical exam for \texttt{DEV 3} is made up of only two, highly structured open questions.

\paragraph{Question 1: } \ \\

\textbf{General shape of the question:} \textit{Given the following class definitions, and a piece of code that uses them, fill in the stack, heap, and PC with all steps taken by the program at runtime.}

\textbf{Concrete example of question:}

\lstset{numbers=left,basicstyle=\ttfamily\small}\lstset{language=[Sharp]C}
\begin{lstlisting}
interface ICounter {
  void Incr(int diff);
}
class Counter : ICounter {
  private int cnt;
  public Counter() {
    this.cnt = 0;
  }
  public void Incr(int diff) {
    this.cnt = (this.cnt + diff);
  }
}
ICounter c = new Counter();
c.Incr(5);
\end{lstlisting}

\textbf{Concrete example of answer:}

{
\small
\begin{enumerate}
\item Stack: 
\begin{tabular}{ |c| }
		\hline
		PC \\
		\hline
		1 \\
		\hline
		
\end{tabular}
	\\

\item Stack: \begin{tabular}{ |c| }
		\hline
		PC \\
		\hline
		13 \\
		\hline
		
	\end{tabular}

Heap: \begin{tabular}{ |c| }
		\hline
		1 \\
		\hline
		\begin{tabular}{c} cnt= \end{tabular} \\
		\hline
		
	\end{tabular}\\

\item Stack: \begin{tabular}{ |c|c|>{\columncolor[gray]{0.8}}c|c|c|c| }
		\hline
		PC & ... & & PC & ret & this \\
		\hline
		13 & ...
		& & 7 & null & ref 1 \\
		\hline
		
	\end{tabular}

Heap: 
\begin{tabular}{ |c| }
		\hline
		1 \\
		\hline
		\begin{tabular}{c} cnt= \end{tabular} \\
		\hline
		
	\end{tabular}\\

\item Stack: \begin{tabular}{ |c|c|>{\columncolor[gray]{0.8}}c|c|c| }
		\hline
		PC & ... & & PC & ret \\
		\hline
		13 & ...
		& & 7 & ref 1 \\
		\hline
		
	\end{tabular}
	\\Heap: \begin{tabular}{ |c| }
		\hline
		1 \\
		\hline
		\begin{tabular}{c} cnt=0 \end{tabular} \\
		\hline
		
	\end{tabular}\\

\item Stack: \begin{tabular}{ |c|c| }
		\hline
		PC & c \\
		\hline
		14 & ref 1 \\
		\hline
		
	\end{tabular}
	\\Heap: \begin{tabular}{ |c| }
		\hline
		1 \\
		\hline
		\begin{tabular}{c} cnt=0 \end{tabular} \\
		\hline
		
	\end{tabular}\\

\item Stack: \begin{tabular}{ |c|c|>{\columncolor[gray]{0.8}}c|c|c|c|c| }
		\hline
		PC & ... & & PC & ret & diff & this \\
		\hline
		14 & ...
		& & 10 & null & 5 & ref 1 \\
		\hline
		
	\end{tabular}
	\\Heap: \begin{tabular}{ |c| }
		\hline
		1 \\
		\hline
		\begin{tabular}{c} cnt=0 \end{tabular} \\
		\hline
		
	\end{tabular}\\

\item Stack: \begin{tabular}{ |c|c|>{\columncolor[gray]{0.8}}c|c|c| }
		\hline
		PC & ... & & PC & ret \\
		\hline
		14 & ...
		& & 10 & ref 1 \\
		\hline
		
	\end{tabular}
	\\Heap: \begin{tabular}{ |c| }
		\hline
		1 \\
		\hline
		\begin{tabular}{c} cnt=5 \end{tabular} \\
		\hline
		
	\end{tabular}\\

\item Stack: \begin{tabular}{ |c|c| }
		\hline
		PC & c \\
		\hline
		15 & ref 1 \\
		\hline
		
	\end{tabular}
	\\Heap: \begin{tabular}{ |c| }
		\hline
		1 \\
		\hline
		\begin{tabular}{c} cnt=5 \end{tabular} \\
		\hline
		
	\end{tabular}

\end{enumerate}
}

\textbf{Points:} \textit{4 (50\% of total).}

\textbf{Grading:} \textit{Full points for more than 90\% of correct names and values. Three points if at least all names are correctly placed on the stack with at least half the values correct. Half points for more than 40\% of correct names and values. Zero points otherwise.}

\textbf{Associated learning objective:} \glsfirst{abs}

\ \\ 

\paragraph{Question 2: } \ \\

\textbf{General shape of question:} \textit{Given the following class definitions, and a piece of code that uses them, fill in the declarations, class definitions, and PC with all steps taken by the compiler while type checking.}

\textbf{Concrete example of question:} 

\lstset{numbers=left,basicstyle=\ttfamily\small}\lstset{language=[Sharp]C}
\begin{lstlisting}
interface ICounter {
  void Incr(int diff);
}
class Counter : ICounter {
  private int cnt;
  public Counter() {
    this.cnt = 0;
  }
  public void Incr(int diff) {
    this.cnt = (this.cnt + diff);
  }
}
ICounter c = new Counter();
c.Incr(5);
\end{lstlisting}

\textbf{Concrete example of answer:} \textit{}

{
	\small
	\begin{enumerate}
\item Declarations: \begin{tabular}{ |c| }
	\hline
	PC \\
	\hline
	1 \\
	\hline
	
\end{tabular}

\item Declarations: \begin{tabular}{ |c| }
		\hline
		PC \\
		\hline
		4 \\
		\hline
		
	\end{tabular}
	\\Classes: \begin{tabular}{ |c|c|c|c|c|c|c|c| }
		\hline
		ICounter \\
		\hline
		\begin{tabular}{c} Incr=(ICounter$\times$int) $\rightarrow$ void \end{tabular} \\
		\hline
		
	\end{tabular}

\item Declarations: \begin{tabular}{ |c|c| }
		\hline
		PC & this \\
		\hline
		7 & Counter \\
		\hline
		
	\end{tabular}
	\\Classes: \begin{tabular}{ |c|c|c|c|c|c|c|c|c|c|c|c|c|c|c|c|c|c| }
		\hline
		Counter & ICounter \\
		\hline
		\begin{tabular}{c} Counter=Counter $\rightarrow$ Counter \\Incr=(Counter$\times$int) $\rightarrow$ void \\cnt=int \end{tabular} & \begin{tabular}{c} Incr=(ICounter$\times$int)$\rightarrow$ void \end{tabular} \\
		\hline
		
	\end{tabular}

\item Declarations: \begin{tabular}{ |c|c|c| }
		\hline
		PC & diff & this \\
		\hline
		9 & int & Counter \\
		\hline
		
	\end{tabular}
	\\Classes: \begin{tabular}{ |c|c|c|c|c|c|c|c|c|c|c|c|c|c|c|c|c|c| }
		\hline
		Counter & ICounter \\
		\hline
		\begin{tabular}{c} Counter=Counter $\rightarrow$ Counter \\Incr=(Counter$\times$int) $\rightarrow$ void \\cnt=int \end{tabular} & \begin{tabular}{c} Incr=(ICounter$\times$int) $\rightarrow$ void \end{tabular} \\
		\hline
		
	\end{tabular}

\item Declarations: \begin{tabular}{ |c| }
		\hline
		PC \\
		\hline
		13 \\
		\hline
		
	\end{tabular}
	\\Classes: \begin{tabular}{ |c|c|c|c|c|c|c|c|c|c|c|c|c|c|c|c|c|c| }
		\hline
		Counter & ICounter \\
		\hline
		\begin{tabular}{c} Counter=Counter $\rightarrow$ Counter \\Incr=(Counter$\times$int) $\rightarrow$ void \\cnt=int \end{tabular} & \begin{tabular}{c} Incr=(ICounter$\times$int) $\rightarrow$ void \end{tabular} \\
		\hline
		
	\end{tabular}

\item Declarations: \begin{tabular}{ |c|c| }
		\hline
		PC & c \\
		\hline
		14 & ICounter \\
		\hline
		
	\end{tabular}
	\\Classes: \begin{tabular}{ |c|c|c|c|c|c|c|c|c|c|c|c|c|c|c|c|c|c| }
		\hline
		Counter & ICounter \\
		\hline
		\begin{tabular}{c} Counter=Counter $\rightarrow$ Counter \\Incr=(Counter$\times$int) $\rightarrow$ void \\cnt=int \end{tabular} & \begin{tabular}{c} Incr=(ICounter$\times$int) $\rightarrow$ void \end{tabular} \\
		\hline
		
	\end{tabular}

\item Declarations: \begin{tabular}{ |c|>{\columncolor[gray]{0.8}}c|c|c|c|c| }
		\hline
		c & & PC & ret & $arg_1$ & this \\
		\hline
		ICounter & & 14 & null & int & Counter \\
		\hline
		
	\end{tabular}
	\\Classes: \begin{tabular}{ |c|c|c|c|c|c|c|c|c|c|c|c|c|c|c|c|c|c| }
		\hline
		Counter & ICounter \\
		\hline
		\begin{tabular}{c} Counter=Counter $\rightarrow$ Counter \\Incr=(Counter$\times$int) $\rightarrow$ void \\cnt=int \end{tabular} & \begin{tabular}{c} Incr=(ICounter$\times$int) $\rightarrow$ void \end{tabular} \\
		\hline
		
	\end{tabular}

\item Declarations: \begin{tabular}{ |c|>{\columncolor[gray]{0.8}}c|c|c|c|c| }
		\hline
		c & & PC & ret & $arg_1$ & this \\
		\hline
		ICounter & & 14 & void & int & Counter \\
		\hline
		
	\end{tabular}
	\\Classes: \begin{tabular}{ |c|c|c|c|c|c|c|c|c|c|c|c|c|c|c|c|c|c| }
		\hline
		Counter & ICounter \\
		\hline
		\begin{tabular}{c} Counter=Counter $\rightarrow$ Counter \\Incr=(Counter$\times$int) $\rightarrow$ void \\cnt=int \end{tabular} & \begin{tabular}{c} Incr=(ICounter$\times$int) $\rightarrow$ void \end{tabular} \\
		\hline
		
	\end{tabular}

\item Declarations: \begin{tabular}{ |c|c| }
		\hline
		PC & c \\
		\hline
		15 & ICounter \\
		\hline
		
	\end{tabular}
	\\Classes: \begin{tabular}{ |c|c|c|c|c|c|c|c|c|c|c|c|c|c|c|c|c|c| }
		\hline
		Counter & ICounter \\
		\hline
		\begin{tabular}{c} Counter=Counter $\rightarrow$ Counter \\Incr=(Counter$\times$int) $\rightarrow$ void \\cnt=int \end{tabular} & \begin{tabular}{c} Incr=(ICounter$\times$int) $\rightarrow$ void \end{tabular} \\
		\hline
		
	\end{tabular}
	
	\end{enumerate}
}

\textbf{Points:} \textit{4 (50\% of total).}

\textbf{Grading:} \textit{Full points for more than 90\% of correct names and types. Three points if at least all names are correctly placed on the declarations and classes, with at least half the types correct. Half points for more than 40\% of correct names and types. Zero points otherwise.}

\textbf{Associated learning objective:} \glsfirst{type}

\ \\
