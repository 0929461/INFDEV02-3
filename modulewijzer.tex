\documentclass[12pt,a4paper,draft]{article}
\usepackage[latin1]{inputenc}
\usepackage{amsmath}
\usepackage{amsfonts}
\usepackage{amssymb}
\usepackage{graphicx}
\author{INFDEV02-3}
\title{The DEV team}

\begin{document}

	\maketitle
	
	\section{Lectures and homework}
	
		\subsection{Week 1 - statically typed programming languages}
		
			\paragraph*{Topics}
			
				\begin{itemize}
					\item What are types?
					\item (\textbf{Advanced}) Typing and semantic rules: how do we read them?
					\item Introduction to Java and C\# (\textbf{advanced}) with type rules and semantics
					\begin{itemize}
						\item Classes
						\item Fields/attributes
						\item Constructor(s), methods, and static methods
						\item Statements, expressions, and primitive types
						\item (\textbf{Advanced}) Lambda's
					\end{itemize}
				\end{itemize}
				
			\paragraph*{Homework}
				\begin{itemize}
					\item Write an example of Python code that would cause a type error in Java/C\#
					\item Given the following semantic and typing rules, write down how we read them; make an example code that uses them
					\item Write a Java/C\# program featuring
					\begin{itemize}
						\item A \texttt{Counter} class;
						\item With a \texttt{count} integer attribute;
						\item With an empty (parameterless) constructor;
						\item With a method \texttt{Reset};
						\item With a method \texttt{Tick};
						\item (\textbf{Advanced}) With a static method/overloaded operator \texttt{Plus} which adds two counters into one;
						\item (\textbf{Advanced}) With a method \texttt{OnTarget} that takes as input a lambda function which will be fired when the counter reaches a given count.
					\end{itemize}
				\end{itemize}
		
		
		\subsection{Week 2 - reuse through polymorphism and generics}
		
		\paragraph*{Topics}
		
		\begin{itemize}
			\item What is code reuse?
			\item Interfaces and implementation
			\item Implicit vs explicit conversion
			\item (\textbf{Advanced}) Implicit and explicit conversion type rules
			\item Runtime type testing
			\item (\textbf{Advanced}) Generic parameters
			\item (\textbf{Advanced}) Interfaces and implementation in the presence of generic parameters
			\item (\textbf{Advanced}) Covariance and contravariance in the presence of generic parameters
		\end{itemize}
		
		\paragraph*{Homework}
		\begin{itemize}
			\item Write a \texttt{Vehicle} interface with a method \texttt{move} and a method \texttt{loadFuel}; \texttt{loadFuel} accepts a \texttt{Fuel} instance, where \texttt{Fuel} is an interface of your writing; \texttt{move} returns a boolean which is \texttt{true} if there is enough fuel, and \texttt{false} otherwise
			\item Write a concrete class \texttt{Car} and a concrete class \texttt{Gasoline} that implement, respectively, \texttt{Vehicle} and \texttt{Fuel}; the \texttt{Car} checks that the given fuel is indeed \texttt{Gasoline}
			\item Write a concrete class \texttt{Truck} and a concrete class \texttt{Diesel} that implement, respectively, \texttt{Vehicle} and \texttt{Fuel}; the \texttt{Truck} checks that the given fuel is indeed \texttt{Diesel}
			\item Write a concrete class \texttt{Enterprise} and a concrete class \texttt{Dilithium} that implement, respectively, \texttt{Vehicle} and \texttt{Fuel}; the \texttt{Enterprise} checks that the given fuel is indeed \texttt{Dilithium}
			\item Make a program that receives three vehicles, without knowing their concrete type, and moves them (without resorting to conversions) until their fuel is up

			\item (\textbf{Advanced}) Make a \texttt{List<T>} interface with methods \texttt{Length}, \texttt{Iterate}, \texttt{Map}, and \texttt{Filter}
			\item (\textbf{Advanced}) Define the concrete classes \texttt{Node<T>} and \texttt{Empty<T>} both implementing \texttt{List<T>}
			\item (\textbf{Advanced}) Make a \texttt{List<Vehicle>}, fill it with a series of concrete vehicles, and make them all move ten times
		\end{itemize}

\end{document}
